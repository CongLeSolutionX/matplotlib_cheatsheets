\documentclass[10pt,landscape,a4paper]{article}
\usepackage[right=10mm, left=10mm, top=10mm, bottom=10mm]{geometry}
\usepackage[utf8]{inputenc}
\usepackage[T1]{fontenc}
\usepackage[english]{babel}
\usepackage[rm,light]{roboto}
\usepackage{xcolor}
\usepackage{graphicx}
\graphicspath{{./figures/}}
\usepackage{multicol}
\usepackage{colortbl}
\usepackage{array}
\setlength\parindent{0pt}
\setlength{\tabcolsep}{2pt}
\baselineskip=0pt
\setlength\columnsep{1em}
\definecolor{Gray}{gray}{0.85}

% --- Listing -----------------------------------------------------------------
\usepackage{listings}
\lstset{
  frame=tb, framesep=4pt, framerule=0pt,
  backgroundcolor=\color{black!5},
  basicstyle=\ttfamily\footnotesize,
  commentstyle=\ttfamily\color{black!50},
  breakatwhitespace=false,
  breaklines=true,
  extendedchars=true,
  keepspaces=true,
  language=Python,
  rulecolor=\color{black},
  showspaces=false,
  showstringspaces=false,
  showtabs=false,
  tabsize=2,
  %
  emph = {
    plot, scatter, imshow, bar, contourf, pie, subplots, spines,
    add_gridspec, add_subplot, set_xscale, set_minor_locator, linestyle,
    dash_capstyle, projection, Stroke, Normal, add_axes, label, savefig,
    get_cmap, histtype, annotate, set_minor_formatter, tick_params,
    fill_betweenx, text, legend, errorbar, boxplot, hist, title, xlabel,
    ylabel, suptitle, fraction, pad, set_fontname, get_xticklabels},
  emphstyle = {\ttfamily\bfseries}
}

% --- Fonts -------------------------------------------------------------------
\usepackage{fontspec}
\usepackage[babel=true]{microtype}
\defaultfontfeatures{Ligatures = TeX, Mapping = tex-text}
\setsansfont{Roboto} [ Path           = fonts/roboto/Roboto-,
                       Extension      = .ttf,
                       UprightFont    = Light,
                       ItalicFont     = LightItalic,
                       BoldFont       = Regular,
                       BoldItalicFont = Italic ]
\setromanfont{RobotoSlab} [ Path           = fonts/roboto-slab/RobotoSlab-,
                            Extension      = .ttf,
                            UprightFont    = Light,
                            BoldFont       = Bold ]
\setmonofont{RobotoMono} [ Path           = fonts/roboto-mono/RobotoMono-,
                           Extension      = .ttf,
                           Scale          = 0.90,
                           UprightFont    = Light,
                           ItalicFont     = LightItalic,
                           BoldFont       = Regular,
                           BoldItalicFont = Italic ]
\renewcommand{\familydefault}{\sfdefault}

% -----------------------------------------------------------------------------
\begin{document}
\thispagestyle{empty}

\section*{\LARGE \rmfamily
          Matplotlib \textcolor{orange}{\mdseries tips \& tricks}}

\begin{multicols*}{3}


% -----------------------------------------------------------------------------
\subsection*{\rmfamily Transparency}

Scatter plots can be enhanced by using transparency (alpha) in order
to show area with higher density. Multiple scatter plots can be
used to delineate a frontier.

\begin{tabular}{@{}m{.774\linewidth}m{.216\linewidth}}
\begin{lstlisting}[belowskip=-\baselineskip]
 X = np.random.normal(-1, 1, 500)
 Y = np.random.normal(-1, 1, 500)
 ax.scatter(X, Y, 50, "0.0", lw=2) # optional
 ax.scatter(X, Y, 50, "1.0", lw=0) # optional
 ax.scatter(X, Y, 40, "C1",  lw=0, alpha=0.1)
\end{lstlisting} &
\raisebox{-0.75em}{\includegraphics[width=\linewidth]{tip-transparency.pdf}}
\end{tabular}

% -----------------------------------------------------------------------------
\subsection*{\rmfamily Rasterization}
If your figure has many graphical elements, such as a huge
scatter, you can rasterize them to save memory and keep other elements
in vector format.
\begin{lstlisting}
 X = np.random.normal(-1, 1, 10_000)
 Y = np.random.normal(-1, 1, 10_000)
 ax.scatter(X, Y, rasterized=True)
 fig.savefig("rasterized-figure.pdf", dpi=600)
\end{lstlisting}

% -----------------------------------------------------------------------------
\subsection*{\rmfamily Offline rendering}

Use the Agg backend to render a figure directly in an array.
\begin{lstlisting}
 from matplotlib.backends.backend_agg import FigureCanvas
 canvas = FigureCanvas(Figure()))
 ... # draw some stuff
 canvas.draw()
 Z = np.array(canvas.renderer.buffer_rgba())
\end{lstlisting}

% -----------------------------------------------------------------------------
\subsection*{\rmfamily Range of continuous colors}

You can use colormap to pick from a range of continuous colors.

\begin{tabular}{@{}m{.774\linewidth}m{.216\linewidth}}
\begin{lstlisting}[belowskip=-\baselineskip]
 X = np.random.randn(1000, 4)
 cmap = plt.get_cmap("Oranges")
 colors = cmap([0.2, 0.4, 0.6, 0.8])

 ax.hist(X, 2, histtype='bar', color=colors)
\end{lstlisting} &
\raisebox{-0.75em}{\includegraphics[width=\linewidth]{tip-color-range.pdf}}
\end{tabular}

% -----------------------------------------------------------------------------
\subsection*{\rmfamily Text outline}
Use text outline to make text more visible.

\begin{tabular}{@{}m{.774\linewidth}m{.216\linewidth}}
\begin{lstlisting}[belowskip=-\baselineskip]
 import matplotlib.patheffects as fx
 text = ax.text(0.5, 0.1, "Label")
 text.set_path_effects([
   fx.Stroke(linewidth=3, foreground='1.0'),
   fx.Normal()])
\end{lstlisting} &
\raisebox{-0.75em}{\includegraphics[width=\linewidth]{tip-outline.pdf}}
\end{tabular}


% -----------------------------------------------------------------------------
\subsection*{\rmfamily Multiline plot}
You can plot several lines at once using {\em None} as separator.

\begin{lstlisting}
 X,Y = [], []
 for x in np.linspace(0, 10*np.pi, 100):
   X.extend([x, x, None]), Y.extend([0, sin(x), None])
 ax.plot(X, Y, "black")
\end{lstlisting}
\includegraphics[width=\linewidth]{tip-multiline.pdf}

% -----------------------------------------------------------------------------
\subsection*{\rmfamily Dotted lines}
To have rounded dotted lines, use a custom {\ttfamily linestyle} and
modify {\ttfamily dash\_capstyle}.
\begin{lstlisting}
 ax.plot([0, 1], [0, 0], "C1",
        linestyle=(0, (0.01, 1)), dash_capstyle="round")
 ax.plot([0, 1], [1, 1], "C1",
        linestyle=(0, (0.01, 2)), dash_capstyle="round")
\end{lstlisting}
\includegraphics[width=\linewidth]{tip-dotted.pdf}

% -----------------------------------------------------------------------------
\subsection*{\rmfamily Combining axes}
You can use overlaid axes with different projections.

\begin{tabular}{@{}m{.774\linewidth}m{.216\linewidth}}
\begin{lstlisting}[belowskip=-\baselineskip]
 ax1 = fig.add_axes([0, 0, 1, 1],
                    label="cartesian")
 ax2 = fig.add_axes([0, 0, 1, 1],
                    label="polar",
                    projection="polar")
\end{lstlisting} &
\raisebox{-0.75em}{\includegraphics[width=\linewidth]{tip-dual-axis.pdf}}
\end{tabular}

\subsection*{\rmfamily Colorbar adjustment}
You can adjust a colorbar's size when adding it.

\begin{tabular}{@{}m{.754\linewidth}m{.236\linewidth}}
\begin{lstlisting}[belowskip=-\baselineskip]
 im = ax.imshow(Z)

 cb = plt.colorbar(im,
         fraction=0.046, pad=0.04)
 cb.set_ticks([])
\end{lstlisting} &
\raisebox{-0.75em}{\includegraphics[width=\linewidth]{tip-colorbar.pdf}}
\end{tabular}

\subsection*{\rmfamily Taking advantage of typography}
You can use a condensed font such as Roboto
Condensed to save space on tick labels.
\begin{lstlisting}
 for tick in ax.get_xticklabels(which='both'):
     tick.set_fontname("Roboto Condensed")
\end{lstlisting}
\includegraphics[width=\linewidth]{tip-font-family.pdf}

\subsection*{\rmfamily Getting rid of margins}
Once your figure is finished, you can call {\ttfamily tight\_layout()}
to remove white margins. If there are remaining margins, you can use
the {\ttfamily pdfcrop} utility (comes with TeX live).


\subsection*{\rmfamily Hatching}
You can achieve a nice visual effect with thick hatch patterns.

\begin{tabular}{@{}m{.774\linewidth}m{.216\linewidth}}
\begin{lstlisting}[belowskip=-\baselineskip]
 cmap = plt.get_cmap("Oranges")
 plt.rcParams['hatch.color'] = cmap(0.2)
 plt.rcParams['hatch.linewidth'] = 8
 ax.bar(X, Y, color=cmap(0.6), hatch="/")
\end{lstlisting} &
\raisebox{-0.75em}{\includegraphics[width=\linewidth]{tip-hatched.pdf}}
\end{tabular}


\subsection*{\rmfamily Read the documentation}

Matplotlib comes with an extensive documentation explaining the
details of each command and is generally accompanied by examples.
Together with the huge online gallery, this documentation is a
gold-mine.

\vfill
%
{\scriptsize
  Matplotlib 3.8.2 handout for tips \& tricks.
  Copyright (c) 2021 Matplotlib Development Team.
  Released under a CC-BY 4.0 International License.
  Supported by NumFOCUS.
\par}



\end{multicols*}
\end{document}
